%!TEX root = uiuc_2018_invitational.tex
\begin{problem}{Impossible Task}
{stdin}{stdout}
{1 second}{}{}

Spring break is already over. Even though feeling that a one week break is far from enough (as usual), 
you know it is time to resume working. Suddenly, you realize that there is a midterm in Number Theory
coming up next week. As skimming through all the practice questions given by your professor (which 
you should have done over break), you realize that there is one type of question where you have no idea
how to solve. The question is of the form: Given a number $n$, calculate the sum of all divisors of $n$ except
$n$ itself. Even worse, since you never went to any lecture this semester (how surprising), you have no
clue where to begin. However, lucky for you, the professor promised that the given $n$ on the exam would
belong to a certain set of number. Thus, as long as you can memorize all the possible results, you will be
good. Since the size of the given set is rather large, you want to write a program to compute all the answers before
starting memorizing.

\InputFile

The first line contains an integer $q \leq 10^6$ which denotes the number of possible $n$ will be given on the exam. The second line contains $a_1$, the first question asked, with $1 \le a_1 \le 10^6$. The rest of questions will be generated by the following rule: 
\begin{align*}
	a_{i+1} &= (a_i^2 \text{ mod } 999983) + 1 & \forall i \in \{1 \hdots q-1\}
\end{align*}

\OutputFile

The output has one line with a single integer, which is the sum of answers to questions with $n = a_i$ for all $1 \le i \le q$. 

\Examples

\begin{example}
\exmp{
3
2
}{%
18
}%
\end{example}

\Notes

In the sample test case, question $n = 2$ has answer $1$; question $n = 5$ has answer $1$; question $26$ has answer $16$. Outputted answer should be $1 + 1 + 16 = 18$. 

\end{problem}
