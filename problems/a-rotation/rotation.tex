%!TEX root = ../uiuc_2017_tryout_1.tex
\begin{problem}{A Simple Geometry Problem}
{stdin}{stdout}
{1 second}{}{}

Rotation plays an important in computational geometry. 
In this problem, we are interested in the following 2-dimensional rotation problem:

\begin{center}
    Given a 2-d point $(x, y)$, what's the position after rotating counterclockwise by an angle $\theta$?
\end{center}

It is perfect if you already know how to solve this problem from other classes like linear algebra.
Otherwise, we will derive the solution as follows.

My favorite derivation is based on the complex numbers. 
A complex number is a number that can be expressed in the form $a + bi$, where $a$ and $b$ are real numbers, and $i$ is a solution of the equation $z^2 = -1$, which is called an imaginary number because there is no real number that satisfies this equation. 
The rotation is actually equivalent to $(x+yi) \cdot (\cos\theta + \sin\theta i) = (x \cos \theta - y \sin \theta) + (x \sin \theta + y \cos \theta) i$.
Therefore, the new $x$ coordinate is $x \cos \theta - y \sin \theta$ and the new $y$ coordinate is $x \sin \theta + y \cos \theta$.
Now, you should be able to solve this problem using the built-in math functions :). Enjoy!

\InputFile

The first line of input contains a positive integer $T$ ($1 \le T \le 100$), which represents the number of test cases in total.
In the follow $T$ lines, every line has three floating numbers, $x$, $y$, and $\theta$ ($0 \le |x|, |y| \le 100, 0 \le \theta \le 360$), as described before. 
$\theta$ is in degrees.
There are exactly 2 digits after the decimal point for all floating numbers.

\OutputFile

For each test case, print a line of the rotated position. All floating numbers are \textbf{rounded up to 2 digits after decimal points}. 

\Examples

\begin{example}
\exmp{
3
1 0 0.00
2 0 90.00
2 1 30.00
}{%
1.00 0.00
0.00 2.00
1.23 1.87
}%
\end{example}

\end{problem}
