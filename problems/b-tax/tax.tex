%!TEX root = uiuc_2018_invitational.tex
\begin{problem}{Barfoosia Tax Return}
{stdin}{stdout}
{1 second}{}{}

It's tax return season in the United States of Barfoosia! Because their IRS employees are lazy, U.S.B's tax system is much simpler: when payroll is delivered, a 50\% tax is withheld, regardless of your pre-tax income or any other personal factors. In the tax return process, a final tax amount will be calculated and a balance will be refunded or collected. The final tax amount will be calculated by a staggered rate based on one's pre-tax annual income: 
\begin{center}
	The part of income falling within \$0-\$10,000 will be tax-free \\
	The part of income falling within \$10,000-\$20,000 will be taxed 10\% \\
	The part of income falling within \$20,000-\$40,000 will be taxed 20\% \\
	The part of income falling within \$40,000-\$80,000 will be taxed 40\% \\
	The part of income above \$80,000 will be taxed 80\% 
\end{center}

For example, if Alice made \$5,000 last year, all her income will fall into the tax-free column, and she should have been taxed \$0. Since \$2,500 was withheld from her payroll, she should be refunded \$2,500. If Bob made \$35,000 last year, his first \$10,000 income is tax-free, his next \$10,000 income is taxed by 10\%, and his remaining \$15,000 income is taxed by 20\%. He should have been taxed
\begin{align*}
	\$10,000 * 0\% + (\$20,000 - \$10,000) * 10\% + (\$35,000 - \$20,000) * 20\% = \$4,000
\end{align*}
Since \$17,500 was withheld from his payroll, he should be refunded \$13,500. \\

Given the pre-tax annual income (always a multiple of 100) of U.S.B residents, compute amount refunded to or owed by each person. 

\InputFile

The first line of input contains a positive integer $N$ ($1 \le N \le 100$), which represents the number of residents in U.S.B. \\
In the following $N$ lines, the $i$th line has one integer $X_i$ ($0 \le X_i \le 1,000,000$, $X_i$ is guaranteed to be a multiple of 100), the pre-tax annual income of the $i$th resident. 

\OutputFile

For each resident, print a line containing tax refunded or owed. If the person should be refunded tax, print "Refund " followed by the amount to be refunded; if the person owes tax, print "Owe " followed by the amount owed; if neither, print "Clear". All printed amounts should be integers. 

\Examples

\begin{example}
\exmp{
4
5000
35000
0
150000
}{%
Refund 2500
Refund 13500
Clear
Owe 2000
}%
\end{example}

\end{problem}
