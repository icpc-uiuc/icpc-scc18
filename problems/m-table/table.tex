%!TEX root = uiuc_2018_invitational.tex
\begin{problem}{Macrosoft}
{stdin}{stdout}
{1 second}{}{}

Macrosoft is planning to dominate the world, but they can't succeed without 
a super duper awesome deluxe magnificent spreadsheet program! It turns out to be your job to
make one as an employee. 

Your spreadsheet program takes in some user input, and renders it as a table in ASCII characters. 
\begin{center}
TODO: add example here
\end{center}
The borders of the table are represented by '-' and '|', while intersections are represented by '+'. All cells in a column share the same width, with spaces added as padding. It is required that numbers align to the right, while everything else aligns to the left. A number is a string that contains only digits. Commas are used to divide user input into cells in each row. 

\InputFile

The first line contains an integer $N$, the number of rows.
The next $N$ lines contains the content of the table. 

It is guaranteed that $1 \leq N \leq 100$ and each input line contains at most $99$ characters. 
The input lines only contain digits (0-9), lower case letters (a-z) and commas (,). There are no spaces. However, it is completely fine for a line to be empty.

\OutputFile

The table, rendered in ASCII characters according to the specficication.

\Examples

\begin{example}
\exmp{
2
first,line
row,2
}{%
+----------+--------+
|first|line|
+----------+--------+
|row  |   2|
+----------+--------+
}%
\exmp{
1
gwyn,,gwynevere,gwyndolin
}{%
+--------++------------------+------------------+
|gwyn||gwynevere|gwyndolin|
+--------++------------------+------------------+
}%
\exmp{
2
\vspace{5mm}
whereisthelastline
}{%
+------------------------------------+
|                  |
+------------------------------------+
|whereisthelastline|
+------------------------------------+
}%
\end{example}

\Notes

In the third example, there are three lines in the input; the second line is empty. 

\end{problem}
