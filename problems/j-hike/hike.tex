%!TEX root = uiuc_2018_invitational.tex

\begin{problem}{Just A Hike!}
{stdin}{stdout}
{1 second}{}{}

The summer just started and you just spent all night last night playing the famous Creator-Destructor game of Ubihard. 
Thus, you decide to do something more helpful this morning - cleaning the attic. Cleaning the attic, although seems to be
a boring job to do, is in fact your favourite chore at home. While doing the task, you always have a chance to go through 
the old photos of your family and spend time recalling all the good memories you had during your childhood. Suddenly, you 
come accross the old photo of you and your family in your first hike. At this very moment, you decide that your second
mission of the summer (the first one is to finish the very interesting game mentioned above) is to get your family to go
on a hike together one more time.\\

You start to plan for the hike using a map that has the estimated height of the countryside area of your hometown. It can be 
represented as a rectangle grid with $m$ rows and $n$ columns. The cell at the $i_{th}$ row and $j_{th}$ column is denoted 
as cell $(i, j)$ and has the height $h[i, j]$. You call a cell $(i, j)$ a peak if its height is greater than or equal all 
the cell adjacent to it. Two cells are considered adjacent if the share a common edge. Thus, a cell has at most $4$ adjacent 
cells. You believe that any peak in the map represents a real mountain and can be used for hiking. You want to find any peak 
here as the destination for your hike. However, since the map is rather large, it is simply impossible to find a peak manually.
As a Computer Science student, you know this is your moment to shine!

\InputFile

The first line contains two integers $m, n$ ($1 \leq m, n \leq 10^5$) which represent the size of the map. The second line
contains $n$ integers $h[1, 1], h[1, 2], \cdots, h[1, n]$ represent the height of all the cells in the first row of the
map. Since the map is rather large, this is the only thing you get. The rest of the map can be calculated with the formula
    \begin{equation}
       h[i, j] = h[1, j]^i \text{ mod } 1000000007
    \end{equation}
where mod denotes the remainder of the division for $10^9 + 7$.

\OutputFile

Print $2$ numbers $i, j$ denotes the coordinate of any peak $(i, j)$.

\Examples

\begin{example}
\exmp{
2 3
2 3 1
}{%
2 2
}%
\end{example}

Explanation: The full map is
    \begin{equation}
    \begin{matrix}
        2 & 3 & 1\\
        4 & 9 & 1
    \end{matrix}
    \end{equation}
Then, clearly the peak is cell $(2, 2)$ with height $9$.

\end{problem}
